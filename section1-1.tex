\section{基本原理}
\par
大多数程序以某种方式和外界交流,不管是通过一个文件、网络、串口,还是控制终端。有的时候,比如网络通讯,单独的 I/O 操作可能需要很长时间来完成。这对软件开发带来了很大的挑战。

\par
Boost.Asio 提供了一些工具来管理这些长时间的操作,而不需要程序依赖基于线程和显式上锁的并发模型。

\par
Boost.Asio 库意在使用 C++ 进行系统级别编程。特别的,Boost.Asio 强调下列目标:

\begin{itemize}
	\item \textbf{可移植性:} 该库应该支持一系列的常用的操作系统,并在这些操作系统上提供一致的行为。
	\item \textbf{可扩展性:} 该库应该能够支持开发可以扩展到处理成千上万的并发连接的网络应用程序。库在每个操作系统上的实现应该使用最佳的机制,以实现这种可扩展性。
	\item \textbf{效率:} 该库应支持诸如散布-聚集 I/O(scatter-gather I/O)等技术,并允许程序尽量减少数据复制。
	\item \textbf{借鉴已有的 API 模型(例如 BSD 套接字(sockets)API):} BSD 套接字 API 被广泛实现和理解,并且有大量的文献覆盖这一主题。其他编程语言也常常使用类似的网络 API 接口。因此,Boost.Asio 在设计时应尽可能利用现有的实践和接口标准。
	\item \textbf{易用:} 这个库应通过采取工具包(toolkit)而非框架(framework)的方法,来降低新用户的学习门槛。也就是说,它应尽量减少新用户在学习时的初始投入时间,让他们只需了解一些基本规则和指导方针。之后,用户只需理解他们所使用的具体函数即可。
	\item \textbf{进一步抽象的基础:} 这个库应允许开发其他提供更高层次抽象的库。例如,可以基于这个库实现一些常用协议的库,如 HTTP 协议的实现。
\end{itemize}
尽管 Boost.Asio 起初是为了网络出现的,它的异步IO的概念被扩展延伸到了操作系统其他的资源,比如串口、文件描述符等。
