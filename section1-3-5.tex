\subsection{Executors}
每个异步代理都有一个关联的执行器。代理的执行器决定代理的完成处理程序如何排队并最终运行。

执行器的示例用法包括:
\begin{itemize}
	\item 协调一组在共享数据结构上运行的异步代理,确保代理的完成处理程序永远不会并发运行(In Boost.Asio, this kind of executor is called a strand.)
	\item 确保代理在靠近数据或事件源(如NIC)的指定执行资源(如CPU)上运行。
	\item 表示一组相关的代理,从而使动态线程池能够做出更明智的调度决策(例如将代理作为一个单元在执行资源之间移动)。
	\item 将所有完成处理程序排队在GUI应用程序线程上运行,以便它们可以安全地更新用户界面元素。
	\item 按原样返回异步操作的默认执行器,以尽可能靠近触发操作完成的事件运行完成处理程序。
	\item 调整异步操作的默认执行器,使其在每个完成处理程序之前和之后运行代码,如日志记录、用户授权或异常处理。
	\item 为异步代理及其完成处理程序指定优先级。
\end{itemize}

异步代理中的异步操作使用代理的关联执行器来:
\begin{itemize}
	\item 在操作未完成时跟踪异步操作所代表的工作的存在。
	\item 取消完成处理程序的队列,以便在操作完成时执行。
	\item 如果这样做可能会导致无意的递归和堆栈溢出,请确保完成处理程序不会重新运行。
\end{itemize}
因此,异步代理的关联执行器代表了一种策略,即代理应如何、在何处以及何时运行,该策略被指定为组成代理的代码的跨领域关注点。