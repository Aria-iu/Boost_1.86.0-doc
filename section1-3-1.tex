\subsection{Asynchronous Operations}
Asynchronous Operation 是 Boost.Asio 异步模型组成的基本单元。Asynchronous Operations 代表一种工作,它们在后台启动和运行,同时,用户初始化这些工作的代码可以继续做其他事情(不会忙等)。

理论上,异步操作的生命周期可以被描述为下列事件和阶段:
Initiating function 是一个可以被用户调用来开启一个异步操作的函数。

Completion handler 是由用户提供的,move-only的函数对象,它最多会被调用一次,并产生异步操作的结果。Completion handler的调用告诉用户这样的事情:操作完成,并且这个操作的副作用被完成。

Initiating function 和 Completion handler 被合并到用户的代码中,如下所示:

同步操作体现为单个函数,因此具有几个固有的语义属性。异步操作采用了同步操作中的一些语义属性,以促进灵活高效的组合。


同步操作特性:
\begin{itemize}
	\item 当同步操作是泛型时,返回类型是从函数及其参数中确定性地导出的
	\item 如果同步操作需要临时资源(如内存、文件描述符或线程),则在从函数返回之前释放此资源。
\end{itemize}

异步操作等价特性:
\begin{itemize}
	\item 当异步操作是泛型操作时,完成处理程序的参数类型和顺序是从发起函数及其参数中确定性地导出的
	\item 如果异步操作需要临时资源(如内存、文件描述符或线程),则在调用完成处理程序之前释放此资源。
\end{itemize}

后者是异步操作的一个重要属性,因为它允许完成处理程序在不重叠资源使用的情况下启动进一步的异步操作。考虑在链中反复重复相同操作的平凡(相对常见)情况:

通过确保在完成处理程序运行之前释放资源,我们避免了操作链的峰值资源使用量加倍。
