\section{基础 Boost.Asio 剖析}
Boost.Asio 可以有同步或者异步的操作,例如 Sockets 。在使用 Boost.Asio 之前,有一个各种 Boost.Asio 不同部分的概念图是有所帮助的。

作为一个入门示例,让我们来考虑一下当你在一个 Socket 上执行连接操作时会发生什么。我们将从研究同步操作开始。

你的程序至少含有一个IO execution context,就是 boost::asio::io\_context 、 boost::asio::thread\_pool 或者 boost::asio::system\_context 对象。这个IO execution context 代表着你的程序如何连接操作系统的IO服务。


\begin{lstlisting}[frame=shadowbox]
boost::asio::io_context io_context;
\end{lstlisting}

为了执行IO操作,程序需要一个IO object ,比如一个 TCP Socket 。

\begin{lstlisting}[frame=shadowbox]
boost::asio::ip::tcp::socket socket(io_context);
\end{lstlisting}

当一个同步连接操作执行后,下面一系列的事件就会发生:

\begin{itemize}
	\item 程序在socket对象(IO对象)上调用方法来初始化连接操作
	\begin{lstlisting}[frame=shadowbox]
socket.connect(server_port);
	\end{lstlisting}
	\item IO对象进而执行IO 执行上下文
	\item IO执行上下文通过系统调用进行连接操作
	\item 操作系统将返回值返回给IO上下文的调用处
	\item IO上下文将所有错误统一转化为一个 boost::system::error\_code的对象来表示。一个error\_code可以被比较,测试是不是任何bool值。之后,结果就被返回给IO对象
	\item 如果操作失败,IO对象抛出一个异常,类型是boost::system::system\_error。如果初始化代码写成如下形式:
	\begin{lstlisting}[frame=shadowbox]
boost::system::error_code ec;	
socket.connect(server_port,ec);
	\end{lstlisting}
	那么,ec就被设置为这个操作的结果,而不会抛出异常
\end{itemize}

以上是同步连接的操作,当使用异步操作时,会有不同的事件序列。

\begin{itemize}
	\item 程序在socket对象(IO对象)上调用方法来初始化连接操作
	\begin{lstlisting}[frame=shadowbox]
socket.async_connect(server_port,your_completion_handler);
	\end{lstlisting}
	这里的 your\_completion\_handler 是一个函数或者一个有这样签名的函数对象:
	\begin{lstlisting}[frame=shadowbox]
void your_completion_handler(const boost::system::error_code& ec);
	\end{lstlisting}
	具体的签名还要看具体的异步操作
	\item IO对象进而执行IO 执行上下文
	\item IO执行上下文向操作系统发送请求,需要开启一个异步连接
	
	程序等待时间(在同步操作中,程序需要等待操作系统连接操作结束),在异步操作中,不需要等待,调用直接返回。
	
	\item 操作系统通过在一个queue中放置一个结果来提醒连接操作完成,准备被IO执行上下文获取
	\item 当使用 io\_context 作为上下文时,程序必须调用 io\_context::run(),进而保证操作的结果被获取到。调用这个函数时,如果有未完成的异步操作,程序会阻塞,所以最好在第一个异步操作之后就调用它
	\item 在这个 run 函数中,IO执行上下文会将结果从 queue 出队,转换为 error\_code ,再传递给 your\_completion\_handler
\end{itemize}
\newpage
下面给出一个实例,作为执行逻辑的参考
\begin{lstlisting}[frame=shadowbox]
#include <iostream>
#include <boost/asio.hpp>

using boost::asio::ip::tcp;

void on_connect(const boost::system::error_code& error) {
	if (!error) {
		std::cout << "Connected to the server!" << std::endl;
	} else {
		std::cerr << "Failed to connect: " << error.message() << std::endl;
	}
}

void on_write(const boost::system::error_code& error, std::size_t bytes_transferred) {
	if (!error) {
		std::cout << "Message sent! Bytes transferred: " << bytes_transferred << std::endl;
	} else {
		std::cerr << "Failed to send message: " << error.message() << std::endl;
	}
}

int main() {
	try {
		boost::asio::io_context io_context;
		
		// 创建一个TCP socket
		tcp::socket socket(io_context);
		
		// 解析服务器地址
		tcp::resolver resolver(io_context);
		auto endpoints = resolver.resolve("127.0.0.1", "8080");
		
		// 异步连接到服务器
		boost::asio::async_connect(socket, endpoints, on_connect);
		
		// 运行I/O上下文以处理异步操作
		io_context.run();
		
		// 构建要发送的消息
		std::string message = "Hello, Server!";
		
		// 异步发送消息到服务器
		boost::asio::async_write(socket, boost::asio::buffer(message), on_write);
		
		// 再次运行I/O上下文以处理异步写操作
		io_context.run();
	} catch (std::exception& e) {
		std::cerr << "Exception: " << e.what() << std::endl;
	}
	
	return 0;
}
\end{lstlisting}
io\_context.run() 会阻塞异步操作直到完成,这就为总体上的异步操作执行顺序提供了基准。